
% Default to the notebook output style

    


% Inherit from the specified cell style.




    
\documentclass{article}

    
    
    \usepackage{graphicx} % Used to insert images
    \usepackage{adjustbox} % Used to constrain images to a maximum size 
    \usepackage{color} % Allow colors to be defined
    \usepackage{enumerate} % Needed for markdown enumerations to work
    \usepackage{geometry} % Used to adjust the document margins
    \usepackage{amsmath} % Equations
    \usepackage{amssymb} % Equations
    \usepackage{eurosym} % defines \euro
    \usepackage[mathletters]{ucs} % Extended unicode (utf-8) support
    \usepackage[utf8x]{inputenc} % Allow utf-8 characters in the tex document
    \usepackage{fancyvrb} % verbatim replacement that allows latex
    \usepackage{grffile} % extends the file name processing of package graphics 
                         % to support a larger range 
    % The hyperref package gives us a pdf with properly built
    % internal navigation ('pdf bookmarks' for the table of contents,
    % internal cross-reference links, web links for URLs, etc.)
    \usepackage{hyperref}
    \usepackage{longtable} % longtable support required by pandoc >1.10
    \usepackage{booktabs}  % table support for pandoc > 1.12.2
    \usepackage{ulem} % ulem is needed to support strikethroughs (\sout)
    

    
    
    \definecolor{orange}{cmyk}{0,0.4,0.8,0.2}
    \definecolor{darkorange}{rgb}{.71,0.21,0.01}
    \definecolor{darkgreen}{rgb}{.12,.54,.11}
    \definecolor{myteal}{rgb}{.26, .44, .56}
    \definecolor{gray}{gray}{0.45}
    \definecolor{lightgray}{gray}{.95}
    \definecolor{mediumgray}{gray}{.8}
    \definecolor{inputbackground}{rgb}{.95, .95, .85}
    \definecolor{outputbackground}{rgb}{.95, .95, .95}
    \definecolor{traceback}{rgb}{1, .95, .95}
    % ansi colors
    \definecolor{red}{rgb}{.6,0,0}
    \definecolor{green}{rgb}{0,.65,0}
    \definecolor{brown}{rgb}{0.6,0.6,0}
    \definecolor{blue}{rgb}{0,.145,.698}
    \definecolor{purple}{rgb}{.698,.145,.698}
    \definecolor{cyan}{rgb}{0,.698,.698}
    \definecolor{lightgray}{gray}{0.5}
    
    % bright ansi colors
    \definecolor{darkgray}{gray}{0.25}
    \definecolor{lightred}{rgb}{1.0,0.39,0.28}
    \definecolor{lightgreen}{rgb}{0.48,0.99,0.0}
    \definecolor{lightblue}{rgb}{0.53,0.81,0.92}
    \definecolor{lightpurple}{rgb}{0.87,0.63,0.87}
    \definecolor{lightcyan}{rgb}{0.5,1.0,0.83}
    
    % commands and environments needed by pandoc snippets
    % extracted from the output of `pandoc -s`
    \providecommand{\tightlist}{%
      \setlength{\itemsep}{0pt}\setlength{\parskip}{0pt}}
    \DefineVerbatimEnvironment{Highlighting}{Verbatim}{commandchars=\\\{\}}
    % Add ',fontsize=\small' for more characters per line
    \newenvironment{Shaded}{}{}
    \newcommand{\KeywordTok}[1]{\textcolor[rgb]{0.00,0.44,0.13}{\textbf{{#1}}}}
    \newcommand{\DataTypeTok}[1]{\textcolor[rgb]{0.56,0.13,0.00}{{#1}}}
    \newcommand{\DecValTok}[1]{\textcolor[rgb]{0.25,0.63,0.44}{{#1}}}
    \newcommand{\BaseNTok}[1]{\textcolor[rgb]{0.25,0.63,0.44}{{#1}}}
    \newcommand{\FloatTok}[1]{\textcolor[rgb]{0.25,0.63,0.44}{{#1}}}
    \newcommand{\CharTok}[1]{\textcolor[rgb]{0.25,0.44,0.63}{{#1}}}
    \newcommand{\StringTok}[1]{\textcolor[rgb]{0.25,0.44,0.63}{{#1}}}
    \newcommand{\CommentTok}[1]{\textcolor[rgb]{0.38,0.63,0.69}{\textit{{#1}}}}
    \newcommand{\OtherTok}[1]{\textcolor[rgb]{0.00,0.44,0.13}{{#1}}}
    \newcommand{\AlertTok}[1]{\textcolor[rgb]{1.00,0.00,0.00}{\textbf{{#1}}}}
    \newcommand{\FunctionTok}[1]{\textcolor[rgb]{0.02,0.16,0.49}{{#1}}}
    \newcommand{\RegionMarkerTok}[1]{{#1}}
    \newcommand{\ErrorTok}[1]{\textcolor[rgb]{1.00,0.00,0.00}{\textbf{{#1}}}}
    \newcommand{\NormalTok}[1]{{#1}}
    
    % Additional commands for more recent versions of Pandoc
    \newcommand{\ConstantTok}[1]{\textcolor[rgb]{0.53,0.00,0.00}{{#1}}}
    \newcommand{\SpecialCharTok}[1]{\textcolor[rgb]{0.25,0.44,0.63}{{#1}}}
    \newcommand{\VerbatimStringTok}[1]{\textcolor[rgb]{0.25,0.44,0.63}{{#1}}}
    \newcommand{\SpecialStringTok}[1]{\textcolor[rgb]{0.73,0.40,0.53}{{#1}}}
    \newcommand{\ImportTok}[1]{{#1}}
    \newcommand{\DocumentationTok}[1]{\textcolor[rgb]{0.73,0.13,0.13}{\textit{{#1}}}}
    \newcommand{\AnnotationTok}[1]{\textcolor[rgb]{0.38,0.63,0.69}{\textbf{\textit{{#1}}}}}
    \newcommand{\CommentVarTok}[1]{\textcolor[rgb]{0.38,0.63,0.69}{\textbf{\textit{{#1}}}}}
    \newcommand{\VariableTok}[1]{\textcolor[rgb]{0.10,0.09,0.49}{{#1}}}
    \newcommand{\ControlFlowTok}[1]{\textcolor[rgb]{0.00,0.44,0.13}{\textbf{{#1}}}}
    \newcommand{\OperatorTok}[1]{\textcolor[rgb]{0.40,0.40,0.40}{{#1}}}
    \newcommand{\BuiltInTok}[1]{{#1}}
    \newcommand{\ExtensionTok}[1]{{#1}}
    \newcommand{\PreprocessorTok}[1]{\textcolor[rgb]{0.74,0.48,0.00}{{#1}}}
    \newcommand{\AttributeTok}[1]{\textcolor[rgb]{0.49,0.56,0.16}{{#1}}}
    \newcommand{\InformationTok}[1]{\textcolor[rgb]{0.38,0.63,0.69}{\textbf{\textit{{#1}}}}}
    \newcommand{\WarningTok}[1]{\textcolor[rgb]{0.38,0.63,0.69}{\textbf{\textit{{#1}}}}}
    
    
    % Define a nice break command that doesn't care if a line doesn't already
    % exist.
    \def\br{\hspace*{\fill} \\* }
    % Math Jax compatability definitions
    \def\gt{>}
    \def\lt{<}
    % Document parameters
    \title{Pset3}
    
    
    

    % Pygments definitions
    
\makeatletter
\def\PY@reset{\let\PY@it=\relax \let\PY@bf=\relax%
    \let\PY@ul=\relax \let\PY@tc=\relax%
    \let\PY@bc=\relax \let\PY@ff=\relax}
\def\PY@tok#1{\csname PY@tok@#1\endcsname}
\def\PY@toks#1+{\ifx\relax#1\empty\else%
    \PY@tok{#1}\expandafter\PY@toks\fi}
\def\PY@do#1{\PY@bc{\PY@tc{\PY@ul{%
    \PY@it{\PY@bf{\PY@ff{#1}}}}}}}
\def\PY#1#2{\PY@reset\PY@toks#1+\relax+\PY@do{#2}}

\expandafter\def\csname PY@tok@gd\endcsname{\def\PY@tc##1{\textcolor[rgb]{0.63,0.00,0.00}{##1}}}
\expandafter\def\csname PY@tok@gu\endcsname{\let\PY@bf=\textbf\def\PY@tc##1{\textcolor[rgb]{0.50,0.00,0.50}{##1}}}
\expandafter\def\csname PY@tok@gt\endcsname{\def\PY@tc##1{\textcolor[rgb]{0.00,0.27,0.87}{##1}}}
\expandafter\def\csname PY@tok@gs\endcsname{\let\PY@bf=\textbf}
\expandafter\def\csname PY@tok@gr\endcsname{\def\PY@tc##1{\textcolor[rgb]{1.00,0.00,0.00}{##1}}}
\expandafter\def\csname PY@tok@cm\endcsname{\let\PY@it=\textit\def\PY@tc##1{\textcolor[rgb]{0.25,0.50,0.50}{##1}}}
\expandafter\def\csname PY@tok@vg\endcsname{\def\PY@tc##1{\textcolor[rgb]{0.10,0.09,0.49}{##1}}}
\expandafter\def\csname PY@tok@vi\endcsname{\def\PY@tc##1{\textcolor[rgb]{0.10,0.09,0.49}{##1}}}
\expandafter\def\csname PY@tok@mh\endcsname{\def\PY@tc##1{\textcolor[rgb]{0.40,0.40,0.40}{##1}}}
\expandafter\def\csname PY@tok@cs\endcsname{\let\PY@it=\textit\def\PY@tc##1{\textcolor[rgb]{0.25,0.50,0.50}{##1}}}
\expandafter\def\csname PY@tok@ge\endcsname{\let\PY@it=\textit}
\expandafter\def\csname PY@tok@vc\endcsname{\def\PY@tc##1{\textcolor[rgb]{0.10,0.09,0.49}{##1}}}
\expandafter\def\csname PY@tok@il\endcsname{\def\PY@tc##1{\textcolor[rgb]{0.40,0.40,0.40}{##1}}}
\expandafter\def\csname PY@tok@go\endcsname{\def\PY@tc##1{\textcolor[rgb]{0.53,0.53,0.53}{##1}}}
\expandafter\def\csname PY@tok@cp\endcsname{\def\PY@tc##1{\textcolor[rgb]{0.74,0.48,0.00}{##1}}}
\expandafter\def\csname PY@tok@gi\endcsname{\def\PY@tc##1{\textcolor[rgb]{0.00,0.63,0.00}{##1}}}
\expandafter\def\csname PY@tok@gh\endcsname{\let\PY@bf=\textbf\def\PY@tc##1{\textcolor[rgb]{0.00,0.00,0.50}{##1}}}
\expandafter\def\csname PY@tok@ni\endcsname{\let\PY@bf=\textbf\def\PY@tc##1{\textcolor[rgb]{0.60,0.60,0.60}{##1}}}
\expandafter\def\csname PY@tok@nl\endcsname{\def\PY@tc##1{\textcolor[rgb]{0.63,0.63,0.00}{##1}}}
\expandafter\def\csname PY@tok@nn\endcsname{\let\PY@bf=\textbf\def\PY@tc##1{\textcolor[rgb]{0.00,0.00,1.00}{##1}}}
\expandafter\def\csname PY@tok@no\endcsname{\def\PY@tc##1{\textcolor[rgb]{0.53,0.00,0.00}{##1}}}
\expandafter\def\csname PY@tok@na\endcsname{\def\PY@tc##1{\textcolor[rgb]{0.49,0.56,0.16}{##1}}}
\expandafter\def\csname PY@tok@nb\endcsname{\def\PY@tc##1{\textcolor[rgb]{0.00,0.50,0.00}{##1}}}
\expandafter\def\csname PY@tok@nc\endcsname{\let\PY@bf=\textbf\def\PY@tc##1{\textcolor[rgb]{0.00,0.00,1.00}{##1}}}
\expandafter\def\csname PY@tok@nd\endcsname{\def\PY@tc##1{\textcolor[rgb]{0.67,0.13,1.00}{##1}}}
\expandafter\def\csname PY@tok@ne\endcsname{\let\PY@bf=\textbf\def\PY@tc##1{\textcolor[rgb]{0.82,0.25,0.23}{##1}}}
\expandafter\def\csname PY@tok@nf\endcsname{\def\PY@tc##1{\textcolor[rgb]{0.00,0.00,1.00}{##1}}}
\expandafter\def\csname PY@tok@si\endcsname{\let\PY@bf=\textbf\def\PY@tc##1{\textcolor[rgb]{0.73,0.40,0.53}{##1}}}
\expandafter\def\csname PY@tok@s2\endcsname{\def\PY@tc##1{\textcolor[rgb]{0.73,0.13,0.13}{##1}}}
\expandafter\def\csname PY@tok@nt\endcsname{\let\PY@bf=\textbf\def\PY@tc##1{\textcolor[rgb]{0.00,0.50,0.00}{##1}}}
\expandafter\def\csname PY@tok@nv\endcsname{\def\PY@tc##1{\textcolor[rgb]{0.10,0.09,0.49}{##1}}}
\expandafter\def\csname PY@tok@s1\endcsname{\def\PY@tc##1{\textcolor[rgb]{0.73,0.13,0.13}{##1}}}
\expandafter\def\csname PY@tok@ch\endcsname{\let\PY@it=\textit\def\PY@tc##1{\textcolor[rgb]{0.25,0.50,0.50}{##1}}}
\expandafter\def\csname PY@tok@m\endcsname{\def\PY@tc##1{\textcolor[rgb]{0.40,0.40,0.40}{##1}}}
\expandafter\def\csname PY@tok@gp\endcsname{\let\PY@bf=\textbf\def\PY@tc##1{\textcolor[rgb]{0.00,0.00,0.50}{##1}}}
\expandafter\def\csname PY@tok@sh\endcsname{\def\PY@tc##1{\textcolor[rgb]{0.73,0.13,0.13}{##1}}}
\expandafter\def\csname PY@tok@ow\endcsname{\let\PY@bf=\textbf\def\PY@tc##1{\textcolor[rgb]{0.67,0.13,1.00}{##1}}}
\expandafter\def\csname PY@tok@sx\endcsname{\def\PY@tc##1{\textcolor[rgb]{0.00,0.50,0.00}{##1}}}
\expandafter\def\csname PY@tok@bp\endcsname{\def\PY@tc##1{\textcolor[rgb]{0.00,0.50,0.00}{##1}}}
\expandafter\def\csname PY@tok@c1\endcsname{\let\PY@it=\textit\def\PY@tc##1{\textcolor[rgb]{0.25,0.50,0.50}{##1}}}
\expandafter\def\csname PY@tok@o\endcsname{\def\PY@tc##1{\textcolor[rgb]{0.40,0.40,0.40}{##1}}}
\expandafter\def\csname PY@tok@kc\endcsname{\let\PY@bf=\textbf\def\PY@tc##1{\textcolor[rgb]{0.00,0.50,0.00}{##1}}}
\expandafter\def\csname PY@tok@c\endcsname{\let\PY@it=\textit\def\PY@tc##1{\textcolor[rgb]{0.25,0.50,0.50}{##1}}}
\expandafter\def\csname PY@tok@mf\endcsname{\def\PY@tc##1{\textcolor[rgb]{0.40,0.40,0.40}{##1}}}
\expandafter\def\csname PY@tok@err\endcsname{\def\PY@bc##1{\setlength{\fboxsep}{0pt}\fcolorbox[rgb]{1.00,0.00,0.00}{1,1,1}{\strut ##1}}}
\expandafter\def\csname PY@tok@mb\endcsname{\def\PY@tc##1{\textcolor[rgb]{0.40,0.40,0.40}{##1}}}
\expandafter\def\csname PY@tok@ss\endcsname{\def\PY@tc##1{\textcolor[rgb]{0.10,0.09,0.49}{##1}}}
\expandafter\def\csname PY@tok@sr\endcsname{\def\PY@tc##1{\textcolor[rgb]{0.73,0.40,0.53}{##1}}}
\expandafter\def\csname PY@tok@mo\endcsname{\def\PY@tc##1{\textcolor[rgb]{0.40,0.40,0.40}{##1}}}
\expandafter\def\csname PY@tok@kd\endcsname{\let\PY@bf=\textbf\def\PY@tc##1{\textcolor[rgb]{0.00,0.50,0.00}{##1}}}
\expandafter\def\csname PY@tok@mi\endcsname{\def\PY@tc##1{\textcolor[rgb]{0.40,0.40,0.40}{##1}}}
\expandafter\def\csname PY@tok@kn\endcsname{\let\PY@bf=\textbf\def\PY@tc##1{\textcolor[rgb]{0.00,0.50,0.00}{##1}}}
\expandafter\def\csname PY@tok@cpf\endcsname{\let\PY@it=\textit\def\PY@tc##1{\textcolor[rgb]{0.25,0.50,0.50}{##1}}}
\expandafter\def\csname PY@tok@kr\endcsname{\let\PY@bf=\textbf\def\PY@tc##1{\textcolor[rgb]{0.00,0.50,0.00}{##1}}}
\expandafter\def\csname PY@tok@s\endcsname{\def\PY@tc##1{\textcolor[rgb]{0.73,0.13,0.13}{##1}}}
\expandafter\def\csname PY@tok@kp\endcsname{\def\PY@tc##1{\textcolor[rgb]{0.00,0.50,0.00}{##1}}}
\expandafter\def\csname PY@tok@w\endcsname{\def\PY@tc##1{\textcolor[rgb]{0.73,0.73,0.73}{##1}}}
\expandafter\def\csname PY@tok@kt\endcsname{\def\PY@tc##1{\textcolor[rgb]{0.69,0.00,0.25}{##1}}}
\expandafter\def\csname PY@tok@sc\endcsname{\def\PY@tc##1{\textcolor[rgb]{0.73,0.13,0.13}{##1}}}
\expandafter\def\csname PY@tok@sb\endcsname{\def\PY@tc##1{\textcolor[rgb]{0.73,0.13,0.13}{##1}}}
\expandafter\def\csname PY@tok@k\endcsname{\let\PY@bf=\textbf\def\PY@tc##1{\textcolor[rgb]{0.00,0.50,0.00}{##1}}}
\expandafter\def\csname PY@tok@se\endcsname{\let\PY@bf=\textbf\def\PY@tc##1{\textcolor[rgb]{0.73,0.40,0.13}{##1}}}
\expandafter\def\csname PY@tok@sd\endcsname{\let\PY@it=\textit\def\PY@tc##1{\textcolor[rgb]{0.73,0.13,0.13}{##1}}}

\def\PYZbs{\char`\\}
\def\PYZus{\char`\_}
\def\PYZob{\char`\{}
\def\PYZcb{\char`\}}
\def\PYZca{\char`\^}
\def\PYZam{\char`\&}
\def\PYZlt{\char`\<}
\def\PYZgt{\char`\>}
\def\PYZsh{\char`\#}
\def\PYZpc{\char`\%}
\def\PYZdl{\char`\$}
\def\PYZhy{\char`\-}
\def\PYZsq{\char`\'}
\def\PYZdq{\char`\"}
\def\PYZti{\char`\~}
% for compatibility with earlier versions
\def\PYZat{@}
\def\PYZlb{[}
\def\PYZrb{]}
\makeatother


    % Exact colors from NB
    \definecolor{incolor}{rgb}{0.0, 0.0, 0.5}
    \definecolor{outcolor}{rgb}{0.545, 0.0, 0.0}



    
    % Prevent overflowing lines due to hard-to-break entities
    \sloppy 
    % Setup hyperref package
    \hypersetup{
      breaklinks=true,  % so long urls are correctly broken across lines
      colorlinks=true,
      urlcolor=blue,
      linkcolor=darkorange,
      citecolor=darkgreen,
      }
    % Slightly bigger margins than the latex defaults
    
    \geometry{verbose,tmargin=1in,bmargin=1in,lmargin=1in,rmargin=1in}
    
    

    \begin{document}
    
    
    \maketitle
    
    

    
    \section{Table of Contents}\label{table-of-contents}

{1~~}Question 1

{2~~}Question 4

{3~~}Heavy lifting for 2.

    \section{Question 1}\label{question-1}

(25pt + 10pt) In this question, we use the file \texttt{brader.csv}
which contains data from Brader, Valentino and Suhay (2008). The file
includes the following variables for \(n=265\) observations:

\begin{itemize}
\tightlist
\item
  the outcome of interest -- a four-point scale in response to \emph{Do
  you think the number of immigrants from foreign countries should be
  increased or decreased?}
\item
  tone of the story treatment (positive or negative)
\item
  ethnicity of the featured immigrant treatment (Mexican or Russian)
\item
  respondents' age
\item
  respondents' income
\end{itemize}

Consider the following ordered logit model for an ordered outcome
variable with four levels:
\[ \Pr(Y_i \leq j \mid X_i) \ = \ \frac{\exp(\psi_j - X_i^\top\beta)}
            {1 + \exp(\psi_j - X_i^\top\beta)} \] for \(j = 1,2,3,4\)
and \(i = 1,...,n\) where \(\psi_4=\infty\) and
\(X_i = [{\tt tone}_i \ {\tt eth}_i \ {\tt ppage}_i \ {\tt ppincimp}_i]^\top\)
(i.e.~no intercept).

    \textbf{a) (5pt)} Write down the likelihood function.

    \emph{To simplify the notation, the indexs} \(i,j,k\) \emph{are used
consistently,} \(i\) \emph{to iterate over the number of
observations,}\(j\) \emph{to iterate over the number of outcomes, and}
\(k\) * to iterate over the predictors.*

The log-likelihood can be easily seen to be:

\[
l = 
\prod_{i=1}^n 
\frac{\exp(\psi_{Y_i} - X_i^\top\beta)}
     {1 + \exp(\psi_{Y_i} - X_i^\top\beta)} -
\frac{\exp(\psi_{Y_i-1} - X_i^\top\beta)}
     {1 + \exp(\psi_{Y_i-1} - X_i^\top\beta)}
\]

To simplify, from here on \(\phi\) will be the inverse link function and
\(\phi'\) its derivative. This allows us to write the above expression
in a more Matricial form as:

\[
\prod_{i=1}^n \sum_{j=1}^4
\tilde M_{i,j}
\phi\left(\psi_{j} - \sum_{k=1}^m X_{ik}\beta_k\right)
\]

where

\[
\tilde M = MK =  M 
\begin{pmatrix}
  1 &  0 &  0 &  0\\
 -1 &  1 &  0 &  0\\
  0 & -1 &  1 &  0\\
  0 &  0 & -1 &  1\\
\end{pmatrix}
\]

And \(M\) is the OneHot matrix for the \(Y_i\), i.e.~a matrix that has
\(n\) rows, and each row is the row vector with zeros everywhere but at
the position \(j\), where the the observed \(Y\) is \(Y_j\).

In practice, however, we are interested in the log-likelihood, which is:

\[
L = \sum_{i=1}^n \log \sum_{j=1}^4
\tilde M_{i,j}
\phi\left(\psi_{j} - \sum_{k=1}^m X_{ik}\beta_k\right)
\]

Warning: We will use a great deal of abstraction because the functions
are a mess otherwise, here we go:

    \begin{Verbatim}[commandchars=\\\{\}]
{\color{incolor}In [{\color{incolor}6}]:} log\PYZus{}likelihood \PY{o}{=} wrapper\PY{p}{(}
            \PY{k+kr}{function}\PY{p}{(}X\PY{p}{,}Y\PY{p}{,}\PY{k+kp}{beta}\PY{p}{,}psi\PY{p}{,}M\PY{p}{,}phi\PY{p}{,}dphi\PY{p}{)}\PY{p}{\PYZob{}}
                \PY{k+kp}{sum}\PY{p}{(}\PY{k+kp}{log}\PY{p}{(}\PY{k+kp}{rowSums}\PY{p}{(}M\PY{o}{*}phi\PY{p}{)}\PY{p}{)}\PY{p}{)}
            \PY{p}{\PYZcb{}}
        \PY{p}{)}
\end{Verbatim}

    The wrapper function is just an abstaction (see it at the end of the
file, in the annex), that computes the variables \texttt{M},
\texttt{phi}\(=\phi\left(\psi_{j} - \sum_{k=1}^m X_{ik}\beta_k\right)\),
\texttt{dphi}\(=\phi'\left(\psi_{j} - \sum_{k=1}^m X_{ik}\beta_k\right)\).
We use it because everything is easier in those variables. When a
function goes trough the ``wrapper'', its arguments change to become
(X,Y,beta,psi), and the wrapper computes M, phi and dphi.

    \begin{enumerate}
\def\labelenumi{\alph{enumi})}
\setcounter{enumi}{1}
\tightlist
\item
  (10pt) Derive the score functions for \(\beta\) and \(\psi_j\).
\end{enumerate}

    To simplify the computations, let
\(\phi_{ij}=\phi(\psi_{j} - X_i^\top\beta)\), and
\(\phi_{ij}'=\phi'(\psi_{j} - X_i^\top\beta)\). Then we will have:

\[
L = \sum_{i=1}^n \log \sum_{j=1}^4
\tilde M_{i,j}
\phi_{i,j}
\]

Since deriveatives and sums commute freely, we can compute the score
easily:

\[
\frac{\partial L}{\partial \beta_k} = -
\sum_{i=1}^n \left (\sum_{j=0}^4 \tilde M_{i,j}\phi_{i,j} \right )^{-1}
\left (\sum_{j=0}^4
\tilde M_{i,j}\phi_{i,j}'\right )  X_{ik}
\]

in \texttt{R} that is:

    \begin{Verbatim}[commandchars=\\\{\}]
{\color{incolor}In [{\color{incolor}7}]:} beta\PYZus{}score \PY{o}{=}  wrapper\PY{p}{(}
            \PY{k+kr}{function}\PY{p}{(}X\PY{p}{,}Y\PY{p}{,}\PY{k+kp}{beta}\PY{p}{,}psi\PY{p}{,}M\PY{p}{,}phi\PY{p}{,}dphi\PY{p}{)}\PY{p}{\PYZob{}}
                \PY{o}{\PYZhy{}}\PY{p}{(}\PY{k+kp}{rowSums}\PY{p}{(}M\PY{o}{*}dphi\PY{p}{)}\PY{o}{/}\PY{k+kp}{rowSums}\PY{p}{(}M\PY{o}{*}phi\PY{p}{)}\PY{p}{)}\PY{o}{\PYZpc{}*\PYZpc{}}X
            \PY{p}{\PYZcb{}}
        \PY{p}{)}
\end{Verbatim}

    For the \(\psi\), the result is equally immediate:

\[
\frac{\partial L}{\partial \psi_j} = 
\sum_{i=1}^n \left (\sum_{j=0}^4 \tilde M_{i,j}\phi_{i,j} \right )^{-1}
\tilde M_{i,j}\phi_{i,j}'
\]

which in \texttt{R} can be written as:

    \begin{Verbatim}[commandchars=\\\{\}]
{\color{incolor}In [{\color{incolor}95}]:} psi\PYZus{}score \PY{o}{=}  wrapper\PY{p}{(}
             \PY{k+kr}{function}\PY{p}{(}X\PY{p}{,}Y\PY{p}{,}\PY{k+kp}{beta}\PY{p}{,}psi\PY{p}{,}M\PY{p}{,}phi\PY{p}{,}dphi\PY{p}{)}\PY{p}{\PYZob{}}
                 r \PY{o}{=} \PY{p}{(}\PY{k+kp}{rowSums}\PY{p}{(}M\PY{o}{*}phi\PY{p}{)}\PY{o}{*}\PY{o}{*}\PY{l+m}{\PYZhy{}1}\PY{o}{\PYZpc{}*\PYZpc{}}\PY{p}{(}M\PY{o}{*}dphi\PY{p}{)}\PY{p}{)}
                 r\PY{p}{[}\PY{l+m}{1}\PY{o}{:}\PY{p}{(}\PY{k+kp}{length}\PY{p}{(}r\PY{p}{)}\PY{l+m}{\PYZhy{}1}\PY{p}{)}\PY{p}{]}
             \PY{p}{\PYZcb{}}
         \PY{p}{)}
\end{Verbatim}

    (10pt) Using (a) and (b), calculate the maximum likelihood estimates of
\(\beta\) and \(\psi_j\) and their standard errors via the
\texttt{optim} function in R. Confirm your results by comparing them to
outputs from the \texttt{polr} function in the \texttt{MASS} package.

    Load the data:

    \begin{Verbatim}[commandchars=\\\{\}]
{\color{incolor}In [{\color{incolor}96}]:} brader \PY{o}{=} read.csv\PY{p}{(}\PY{l+s}{\PYZsq{}}\PY{l+s}{Data/brader.csv\PYZsq{}}\PY{p}{)}
\end{Verbatim}

    Prepare a Handler function for \texttt{optim}

    \begin{Verbatim}[commandchars=\\\{\}]
{\color{incolor}In [{\color{incolor}104}]:} likelihood\PYZus{}handler \PY{o}{=} \PY{k+kr}{function} \PY{p}{(}x\PY{p}{,}data\PY{p}{)}\PY{p}{\PYZob{}}
              X \PY{o}{=} data\PY{p}{[[}\PY{l+m}{1}\PY{p}{]]}
              Y \PY{o}{=} data\PY{p}{[[}\PY{l+m}{2}\PY{p}{]]}
              o \PY{o}{=} \PY{k+kp}{ncol}\PY{p}{(}X\PY{p}{)}
              beta \PY{o}{=} x\PY{p}{[}\PY{l+m}{1}\PY{o}{:}o\PY{p}{]}
              psi \PY{o}{=} x\PY{p}{[}\PY{p}{(}o\PY{l+m}{+1}\PY{p}{)}\PY{o}{:}\PY{k+kp}{length}\PY{p}{(}x\PY{p}{)}\PY{p}{]}
              l \PY{o}{=} \PY{k+kp}{length}\PY{p}{(}psi\PY{p}{)}
              diff\PYZus{}psi \PY{o}{=} psi\PY{p}{[}\PY{l+m}{2}\PY{o}{:}l\PY{p}{]}\PY{o}{\PYZhy{}}psi\PY{p}{[}\PY{l+m}{1}\PY{o}{:}\PY{p}{(}l\PY{l+m}{\PYZhy{}1}\PY{p}{)}\PY{p}{]}
              \PY{k+kr}{if}\PY{p}{(}\PY{k+kp}{min}\PY{p}{(}diff\PYZus{}psi\PY{p}{)}\PY{o}{\PYZlt{}=}\PY{l+m}{0}\PY{p}{)}
                  \PY{k+kr}{return} \PY{p}{(}\PY{l+m}{1E5}\PY{p}{)}
              \PY{k+kr}{return}\PY{p}{(}\PY{o}{\PYZhy{}}log\PYZus{}likelihood\PY{p}{(}X\PY{p}{,}Y\PY{p}{,}\PY{k+kp}{beta}\PY{p}{,}psi\PY{p}{)}\PY{p}{)}
          \PY{p}{\PYZcb{}}
\end{Verbatim}

    \begin{Verbatim}[commandchars=\\\{\}]
{\color{incolor}In [{\color{incolor}114}]:} gradient\PYZus{}handler \PY{o}{=} \PY{k+kr}{function}\PY{p}{(}x\PY{p}{,}data\PY{p}{)}\PY{p}{\PYZob{}}
              X \PY{o}{=} data\PY{p}{[[}\PY{l+m}{1}\PY{p}{]]}
              Y \PY{o}{=} data\PY{p}{[[}\PY{l+m}{2}\PY{p}{]]}
              o \PY{o}{=} \PY{k+kp}{ncol}\PY{p}{(}X\PY{p}{)}
              beta \PY{o}{=} x\PY{p}{[}\PY{l+m}{1}\PY{o}{:}o\PY{p}{]}
              psi \PY{o}{=} x\PY{p}{[}\PY{p}{(}o\PY{l+m}{+1}\PY{p}{)}\PY{o}{:}\PY{k+kp}{length}\PY{p}{(}x\PY{p}{)}\PY{p}{]}
              l \PY{o}{=} \PY{k+kp}{length}\PY{p}{(}psi\PY{p}{)}
              diff\PYZus{}psi \PY{o}{=} psi\PY{p}{[}\PY{l+m}{2}\PY{o}{:}l\PY{p}{]}\PY{o}{\PYZhy{}}psi\PY{p}{[}\PY{l+m}{1}\PY{o}{:}\PY{p}{(}l\PY{l+m}{\PYZhy{}1}\PY{p}{)}\PY{p}{]}
              \PY{k+kr}{if}\PY{p}{(}\PY{k+kp}{min}\PY{p}{(}diff\PYZus{}psi\PY{p}{)}\PY{o}{\PYZlt{}=}\PY{l+m}{0}\PY{p}{)}
                  \PY{k+kr}{return}\PY{p}{(}x\PY{o}{*}\PY{l+m}{0}\PY{p}{)}\PY{p}{;}
              sbeta \PY{o}{=} \PY{p}{(}\PY{o}{\PYZhy{}}beta\PYZus{}score\PY{p}{(}X\PY{p}{,}Y\PY{p}{,}\PY{k+kp}{beta}\PY{p}{,}psi\PY{p}{)}\PY{p}{)} 
              spsi  \PY{o}{=} \PY{p}{(}\PY{o}{\PYZhy{}}psi\PYZus{}score\PY{p}{(}X\PY{p}{,}Y\PY{p}{,}\PY{k+kp}{beta}\PY{p}{,}psi\PY{p}{)}\PY{p}{)} 
              \PY{k+kp}{append}\PY{p}{(}sbeta\PY{p}{,}spsi\PY{p}{)}
          \PY{p}{\PYZcb{}}
\end{Verbatim}

    Then we prepare the data as our function needs it

    \begin{Verbatim}[commandchars=\\\{\}]
{\color{incolor}In [{\color{incolor}111}]:} Y \PY{o}{=} brader\PY{o}{\PYZdl{}}immigr
          X \PY{o}{=} \PY{k+kp}{data.matrix}\PY{p}{(}brader\PY{p}{)}\PY{p}{[}\PY{p}{,}\PY{l+m}{2}\PY{o}{:}\PY{l+m}{5}\PY{p}{]}
          data \PY{o}{=} \PY{k+kt}{list}\PY{p}{(}X\PY{p}{,}Y\PY{p}{)}
          beta \PY{o}{=} \PY{k+kt}{c}\PY{p}{(}\PY{l+m}{0.5}\PY{p}{,} \PY{l+m}{0}\PY{p}{,} \PY{l+m}{0}\PY{p}{,} \PY{l+m}{0}\PY{p}{)}
          psi \PY{o}{=} \PY{k+kt}{c}\PY{p}{(}\PY{l+m}{\PYZhy{}1}\PY{p}{,} \PY{l+m}{0}\PY{p}{,} \PY{l+m}{1}\PY{p}{)}
          betapsi \PY{o}{=} \PY{k+kp}{append}\PY{p}{(}\PY{k+kp}{beta}\PY{p}{,}psi\PY{p}{)}
          betapsi
\end{Verbatim}

    \begin{enumerate*}
\item 0.5
\item 0
\item 0
\item 0
\item -1
\item 0
\item 1
\end{enumerate*}

    
    \begin{Verbatim}[commandchars=\\\{\}]
{\color{incolor}In [{\color{incolor}112}]:} psi\PYZus{}score\PY{p}{(}X\PY{p}{,}Y\PY{p}{,}\PY{k+kp}{beta}\PY{p}{,}psi\PY{p}{)}
\end{Verbatim}

    \begin{enumerate*}
\item -48.8333052835874
\item -6.56482749873738
\item -11.6548684369225
\end{enumerate*}

    
    And we run the optimization

    \begin{Verbatim}[commandchars=\\\{\}]
{\color{incolor}In [{\color{incolor}113}]:} r \PY{o}{=} optim\PY{p}{(}betapsi\PY{p}{,}likelihood\PYZus{}handler\PY{p}{,} gr \PY{o}{=} gradient\PYZus{}handler\PY{p}{,} data \PY{o}{=} data\PY{p}{,}
                    hessian \PY{o}{=} \PY{l+m}{1}\PY{p}{,} control \PY{o}{=} \PY{p}{(}reltol \PY{o}{=} \PY{l+m}{1E\PYZhy{}12}\PY{p}{)}\PY{p}{)}
          r\PY{o}{\PYZdl{}}par\PY{p}{[}\PY{l+m}{1}\PY{o}{:}\PY{l+m}{4}\PY{p}{]}
          r\PY{o}{\PYZdl{}}par\PY{p}{[}\PY{l+m}{5}\PY{o}{:}\PY{l+m}{7}\PY{p}{]}
          \PY{o}{\PYZhy{}}log\PYZus{}likelihood\PY{p}{(}X\PY{p}{,}Y\PY{p}{,}r\PY{o}{\PYZdl{}}par\PY{p}{[}\PY{l+m}{1}\PY{o}{:}\PY{l+m}{4}\PY{p}{]}\PY{p}{,}r\PY{o}{\PYZdl{}}par\PY{p}{[}\PY{l+m}{5}\PY{o}{:}\PY{l+m}{7}\PY{p}{]}\PY{p}{)}
\end{Verbatim}

    \begin{enumerate*}
\item 0.75057472585271
\item 0.182514633025783
\item 0.00954246157357417
\item 0.00453801085115179
\end{enumerate*}

    
    \begin{enumerate*}
\item -1.64987523554466
\item 0.153712692971585
\item 1.37459144576628
\end{enumerate*}

    
    325.94791194349

    
    \begin{Verbatim}[commandchars=\\\{\}]
{\color{incolor}In [{\color{incolor}31}]:} \PY{k+kn}{library}\PY{p}{(}MASS\PY{p}{)}
         
         plr \PY{o}{\PYZlt{}\PYZhy{}} polr\PY{p}{(} \PY{k+kp}{factor}\PY{p}{(}immigr\PY{p}{)} \PY{o}{\PYZti{}} tone \PY{o}{+} eth \PY{o}{+} ppage \PY{o}{+} ppincimp\PY{p}{,} 
                            data \PY{o}{=} brader\PY{p}{,} method \PY{o}{=} \PY{l+s}{\PYZdq{}}\PY{l+s}{logistic\PYZdq{}}\PY{p}{,} Hess \PY{o}{=} \PY{l+m}{1}\PY{p}{)}
         \PY{k+kp}{summary}\PY{p}{(}plr\PY{p}{)}
         \PY{o}{\PYZhy{}}log\PYZus{}likelihood\PY{p}{(}X\PY{p}{,}Y\PY{p}{,}plr\PY{o}{\PYZdl{}}coefficients\PY{p}{,}plr\PY{o}{\PYZdl{}}z\PY{p}{)}
\end{Verbatim}

    
    \begin{verbatim}
Call:
polr(formula = factor(immigr) ~ tone + eth + ppage + ppincimp, 
    data = brader, Hess = 1, method = "logistic")

Coefficients:
            Value Std. Error t value
tone     0.749446   0.230241  3.2550
eth      0.166225   0.227063  0.7321
ppage    0.009345   0.007131  1.3105
ppincimp 0.004149   0.029511  0.1406

Intercepts:
    Value   Std. Error t value
1|2 -1.6671  0.5664    -2.9434
2|3  0.1318  0.5401     0.2441
3|4  1.3535  0.5466     2.4761

Residual Deviance: 651.8892 
AIC: 665.8892 
    \end{verbatim}

    
    325.944618379387

    
    ** d. (10pt) Bonus question. ** The standard ordered logit model is
sometimes called the proportional odds model because it assumes the
effect of \(X_i\) is constant across levels on the odds ratio scale. One
approach to relax this assumption is to allow the coefficients to vary
across levels, i.e.,
\[ \Pr(Y_i \leq j \mid X_i) \ = \ \frac{\exp(\psi_j - X_i^\top\beta_j)}
            {1 + \exp(\psi_j - X_i^\top\beta_j)} \] for \(j = 1,2,3,4\)
and \(i = 1,...,n\) where \(\beta_4=0\) (i.e.~the fourth group is a
reference group), and \(\psi_4=\infty\). For this model, derive the
likelihood and score functions, and use \texttt{optim} to obtain the
maximum likelihood estimates of \(\beta_j\) and \(\psi_j\) and their
standard errors for the \texttt{brader} data.

    A direct modification of the above exercise, substituting \(\beta_k\)
for \(\beta_{k,j}\) works: We will have:

\[
L = \sum_{i=1}^n \log \sum_{j=1}^4
\tilde M_{i,j}
\phi\left(\psi_{j} - \sum_{k=1}^m X_{ik}\beta_{kj}\right)
\]

therefore, since the formula is essentialy the same (and thanks to
\texttt{R} not really caring much about the shape of the objects --
which is nice!), the same \texttt{log\_likelihood} coded above still
works!

    Since deriveatives and sums commute freely, we can compute the score
easily:

\[
\frac{\partial L}{\partial \beta_{kj}} = -
\sum_{i=1}^n \left (\sum_{j=0}^4 \tilde M_{i,j}\phi_{i,j} \right )^{-1}
\tilde M_{i,j}\phi_{i,j}'  X_{ik}
\]

in this case we must modify the function, to create a new function:

    \begin{Verbatim}[commandchars=\\\{\}]
{\color{incolor}In [{\color{incolor}69}]:} extra\PYZus{}beta\PYZus{}score \PY{o}{=} wrapper\PY{p}{(}
             \PY{k+kr}{function}\PY{p}{(}X\PY{p}{,}Y\PY{p}{,}\PY{k+kp}{beta}\PY{p}{,}psi\PY{p}{,}M\PY{p}{,}phi\PY{p}{,}dphi\PY{p}{)}\PY{p}{\PYZob{}}
                 \PY{c+c1}{\PYZsh{}this corresponds to a matrix that has the same elements in every}
                 \PY{c+c1}{\PYZsh{}row, and are the (sum **)\PYZca{}\PYZob{}\PYZhy{}1\PYZcb{} in the formula above}
                 M1 \PY{o}{=} \PY{k+kt}{matrix}\PY{p}{(}\PY{k+kp}{rep}\PY{p}{(}\PY{k+kp}{rowSums}\PY{p}{(}M\PY{o}{*}phi\PY{p}{)}\PY{p}{,}\PY{k+kp}{ncol}\PY{p}{(}M\PY{p}{)}\PY{p}{)}\PY{p}{,}ncol\PY{o}{=}\PY{k+kp}{ncol}\PY{p}{(}M\PY{p}{)}\PY{p}{,}byrow\PY{o}{=}\PY{l+m}{0}\PY{p}{)}
                 \PY{o}{\PYZhy{}}\PY{k+kp}{t}\PY{p}{(}M\PY{o}{*}dphi\PY{o}{/}M1\PY{p}{)}\PY{o}{\PYZpc{}*\PYZpc{}}X
             \PY{p}{\PYZcb{}}
         \PY{p}{)}
\end{Verbatim}

    For the \(\psi\), the result is equally immediate:

\[
\frac{\partial L}{\partial \psi_j} = 
\sum_{i=1}^n \left (\sum_{j=0}^4 \tilde M_{i,j}\phi_{i,j} \right )^{-1}
\tilde M_{i,j}\phi_{i,j}'
\]

As for the likelihood, the same function does the job

    \section{Question 4}\label{question-4}

Cross Validation for Polynomial Regression. (18 points) Consider the
following four data generating processes:

\begin{itemize}
\tightlist
\item
  DGP 1:
  \(Y = -2* 1_{\{X < -3\}} + 2.55* 1_{\{ X > -2\}} - 2* 1_{\{X>0\}} + 4* 1_{\{X > 2\}} -1* 1_{\{ X > 3\}}+ \epsilon\)
\item
  DGP 2: \(Y = 6 + 0.4 X - 0.36X^2 + 0.005 X^3 + \epsilon\)
\item
  DGP 3: \$Y = 2.83 * \sin(\frac{\pi}{2} \times X) +\epsilon \$
\item
  DGP 4: \(Y = 4 * \sin(3 \pi \times X) * 1_{\{X>0\}}+ \epsilon\)
\end{itemize}

\(X\) is drawn from the uniform distribution in {[}-4,4{]} and
\$\epsilon
 \$ is drawn from a standard normal (\(\mu =0\), \(\sigma^2\) = 1).
\textbackslash{}begin\{enumerate\}

    \begin{Verbatim}[commandchars=\\\{\}]
{\color{incolor}In [{\color{incolor} }]:} DGP1 \PY{o}{=} \PY{k+kr}{function}\PY{p}{(}X\PY{p}{)}\PY{p}{\PYZob{}}
            \PY{l+m}{\PYZhy{}2}   \PY{o}{*}\PY{p}{(}X \PY{o}{\PYZlt{}} \PY{l+m}{\PYZhy{}3}\PY{p}{)}
            \PY{l+m}{+2.55}\PY{o}{*}\PY{p}{(}X \PY{o}{\PYZgt{}} \PY{l+m}{\PYZhy{}2}\PY{p}{)}
            \PY{l+m}{\PYZhy{}2}   \PY{o}{*}\PY{p}{(}X \PY{o}{\PYZgt{}}  \PY{l+m}{0}\PY{p}{)}
            \PY{l+m}{+4}   \PY{o}{*}\PY{p}{(}X \PY{o}{\PYZgt{}}  \PY{l+m}{2}\PY{p}{)}
            \PY{l+m}{\PYZhy{}1}   \PY{o}{*}\PY{p}{(}X \PY{o}{\PYZgt{}}  \PY{l+m}{3}\PY{p}{)}\PY{p}{\PYZcb{}}
        DGP2 \PY{o}{=} \PY{k+kr}{function}\PY{p}{(}X\PY{p}{)}\PY{p}{\PYZob{}}
            \PY{l+m}{6}\PY{l+m}{+0.4}\PY{o}{*}X\PY{l+m}{\PYZhy{}0.36}\PY{o}{*}X\PY{o}{\PYZca{}}\PY{l+m}{2}\PY{l+m}{+0.005}\PY{o}{*}X\PY{o}{\PYZca{}}\PY{l+m}{3}
        \PY{p}{\PYZcb{}}
        DGP3 \PY{o}{=} \PY{k+kr}{function}\PY{p}{(}X\PY{p}{)}\PY{p}{\PYZob{}}
            \PY{l+m}{2.83}\PY{o}{*}\PY{k+kp}{sin}\PY{p}{(}\PY{k+kc}{pi}\PY{o}{/}\PY{l+m}{2}\PY{o}{*}X\PY{p}{)}
        \PY{p}{\PYZcb{}}
        DGP4 \PY{o}{=} \PY{k+kr}{function}\PY{p}{(}X\PY{p}{)}\PY{p}{\PYZob{}}
            \PY{l+m}{4}\PY{o}{*}\PY{k+kp}{sin}\PY{p}{(}\PY{l+m}{3}\PY{o}{*}\PY{k+kc}{pi}\PY{o}{*}x\PY{p}{)}\PY{o}{*}\PY{p}{(}X\PY{o}{\PYZgt{}}\PY{l+m}{0}\PY{p}{)}
        \PY{p}{\PYZcb{}}
        ERR \PY{o}{=} rnorm
        DGP \PY{o}{=} \PY{k+kt}{c}\PY{p}{(}DGP1\PY{p}{,}DGP2\PY{p}{,}DGP3\PY{p}{,}DGP4\PY{p}{)}
\end{Verbatim}

    (5 pts.) Write a function to estimate the generalization error of a
polynomial by \(k\)-fold cross-validation. It should take as arguments
the data, the degree of the polynomial, and the number of folds \(k\).
It should return the cross-validation mean squared error.

    \begin{Verbatim}[commandchars=\\\{\}]
{\color{incolor}In [{\color{incolor} }]:} n \PY{o}{=} \PY{k+kp}{nrow}\PY{p}{(}data\PY{p}{)}\PY{p}{;}
        data \PY{o}{=} data\PY{p}{[}\PY{k+kp}{sample}\PY{p}{(}n\PY{p}{)}\PY{p}{,}\PY{p}{]}
        folds \PY{o}{=} \PY{k+kp}{cut}\PY{p}{(}\PY{k+kp}{seq}\PY{p}{(}\PY{l+m}{1}\PY{p}{,}data\PY{p}{)}\PY{p}{,}breaks\PY{o}{=}N\PY{p}{,}labels\PY{o}{=}\PY{k+kc}{FALSE}\PY{p}{)}
        \PY{k+kr}{for}\PY{p}{(}i \PY{k+kr}{in} \PY{l+m}{1}\PY{o}{:}N\PY{p}{)}
        \PY{p}{\PYZob{}}
            ind \PY{o}{=} \PY{k+kp}{which}\PY{p}{(}folds\PY{o}{==}i\PY{p}{,}arr.ind\PY{o}{=}\PY{k+kc}{TRUE}\PY{p}{)}
            test \PY{o}{=} yourData\PY{p}{[}ind\PY{p}{,}\PY{p}{]}
            train \PY{o}{=} yourData\PY{p}{[}\PY{o}{\PYZhy{}}ind\PY{p}{,}\PY{p}{]}
        \PY{p}{\PYZcb{}}
\end{Verbatim}

    \begin{Verbatim}[commandchars=\\\{\}]
{\color{incolor}In [{\color{incolor} }]:} s \PY{o}{=} \PY{k+kt}{matrix}\PY{p}{(}\PY{k+kt}{c}\PY{p}{(}\PY{l+m}{1}\PY{p}{,}\PY{l+m}{2}\PY{p}{,}\PY{l+m}{3}\PY{p}{,}\PY{l+m}{4}\PY{p}{,}\PY{l+m}{1}\PY{p}{,}\PY{l+m}{2}\PY{p}{,}\PY{l+m}{3}\PY{p}{,}\PY{l+m}{4}\PY{p}{,}\PY{l+m}{1}\PY{p}{,}\PY{l+m}{2}\PY{p}{,}\PY{l+m}{3}\PY{p}{,}\PY{l+m}{4}\PY{p}{)}\PY{p}{,}nrow \PY{o}{=} \PY{l+m}{4}\PY{p}{,}byrow \PY{o}{=} \PY{l+m}{1}\PY{p}{)}
\end{Verbatim}

    \begin{Verbatim}[commandchars=\\\{\}]
{\color{incolor}In [{\color{incolor} }]:} \PY{k+kp}{rowSums}\PY{p}{(}s\PY{p}{)}
\end{Verbatim}

    \begin{Verbatim}[commandchars=\\\{\}]
{\color{incolor}In [{\color{incolor} }]:} s\PY{p}{[}\PY{p}{,}\PY{l+m}{3}\PY{p}{]} \PY{o}{=} \PY{l+m}{1}
\end{Verbatim}

    \section{Heavy lifting for 2.}\label{heavy-lifting-for-2.}

This is where dirty things happen so that the inline code looks
beautiful

    This is the \texttt{wrapper} function used in exercise 2, it precomputes
everything the equations need, and thus it greatly simplifies the
calculations!

    \begin{Verbatim}[commandchars=\\\{\}]
{\color{incolor}In [{\color{incolor}3}]:} wrapper \PY{o}{=} \PY{k+kr}{function}\PY{p}{(}\PY{k+kp}{expression}\PY{p}{)}\PY{p}{\PYZob{}}
            \PY{k+kr}{function}\PY{p}{(}X\PY{p}{,}Y\PY{p}{,}\PY{k+kp}{beta}\PY{p}{,}psi\PY{p}{,}
                     phi \PY{o}{=} \PY{k+kr}{function}\PY{p}{(}x\PY{p}{)}\PY{p}{\PYZob{}}\PY{k+kp}{exp}\PY{p}{(}x\PY{p}{)}\PY{o}{/}\PY{p}{(}\PY{l+m}{1}\PY{o}{+}\PY{k+kp}{exp}\PY{p}{(}x\PY{p}{)}\PY{p}{)}\PY{p}{\PYZcb{}}\PY{p}{,}
                     dphi \PY{o}{=} \PY{k+kr}{function}\PY{p}{(}x\PY{p}{)}\PY{p}{\PYZob{}}\PY{k+kp}{exp}\PY{p}{(}x\PY{p}{)}\PY{o}{/}\PY{p}{(}\PY{l+m}{1}\PY{o}{+}\PY{k+kp}{exp}\PY{p}{(}x\PY{p}{)}\PY{p}{)}\PY{o}{*}\PY{o}{*}\PY{l+m}{2}\PY{p}{\PYZcb{}}\PY{p}{)}
            \PY{p}{\PYZob{}}
                psi \PY{o}{=} \PY{k+kp}{append}\PY{p}{(}psi\PY{p}{,}\PY{l+m}{10}\PY{p}{)}\PY{p}{;} \PY{c+c1}{\PYZsh{}add the \PYZsq{}infinity\PYZsq{}}
                m \PY{o}{=} \PY{k+kp}{length}\PY{p}{(}psi\PY{p}{)}\PY{p}{;}
                M \PY{o}{=} compute\PYZus{}M\PY{p}{(}Y\PY{p}{,}m\PY{p}{)}\PY{p}{;}
                l \PY{o}{=} linear\PYZus{}combination\PY{p}{(}X\PY{p}{,}Y\PY{p}{,}\PY{k+kp}{beta}\PY{p}{,}psi\PY{p}{)}\PY{p}{;}
                
                 phix \PY{o}{=} phi\PY{p}{(}l\PY{p}{)}\PY{p}{;}   phix\PY{p}{[}\PY{p}{,}m\PY{p}{]}\PY{o}{=}\PY{l+m}{1}\PY{p}{;} \PY{c+c1}{\PYZsh{}impose the \PYZsq{}infinity\PYZsq{}}
                dphix \PY{o}{=} phi\PY{p}{(}l\PY{p}{)}\PY{p}{;}  dphix\PY{p}{[}\PY{p}{,}m\PY{p}{]}\PY{o}{=}\PY{l+m}{0}\PY{p}{;} \PY{c+c1}{\PYZsh{}impose the \PYZsq{}infinity\PYZsq{}}
        
                 \PY{c+c1}{\PYZsh{}this is where everything happens}
                \PY{k+kp}{expression}\PY{p}{(}X\PY{p}{,}Y\PY{p}{,}\PY{k+kp}{beta}\PY{p}{,}psi\PY{p}{,}M\PY{p}{,}phix\PY{p}{,}dphix\PY{p}{)}\PY{p}{;}
            \PY{p}{\PYZcb{}}
        \PY{p}{\PYZcb{}}
\end{Verbatim}

    In order for the code to run, we need the \texttt{compute\_M} function
that creates the \texttt{M} matrix defined above:

    \begin{Verbatim}[commandchars=\\\{\}]
{\color{incolor}In [{\color{incolor}4}]:} compute\PYZus{}M \PY{o}{=} \PY{k+kr}{function} \PY{p}{(}Y\PY{p}{,}m\PY{p}{)}\PY{p}{\PYZob{}}
            
            \PY{c+c1}{\PYZsh{}This is just a One Hot encoder:}
            n \PY{o}{=} \PY{k+kp}{length}\PY{p}{(}Y\PY{p}{)}
            M1 \PY{o}{=} \PY{k+kp}{t}\PY{p}{(}\PY{k+kt}{matrix}\PY{p}{(}\PY{k+kp}{rep}\PY{p}{(}\PY{k+kt}{c}\PY{p}{(}\PY{l+m}{1}\PY{o}{:}m\PY{p}{)}\PY{p}{,}n\PY{p}{)}\PY{p}{,}m\PY{p}{)}\PY{p}{)}
            M2 \PY{o}{=} \PY{k+kt}{matrix}\PY{p}{(}\PY{k+kp}{rep}\PY{p}{(}\PY{k+kt}{c}\PY{p}{(}Y\PY{p}{)}\PY{p}{,}m\PY{p}{)}\PY{p}{,}n\PY{p}{)}
            M \PY{o}{=} \PY{p}{(}M1\PY{o}{==}M2\PY{p}{)}\PY{l+m}{+0}
            
            \PY{c+c1}{\PYZsh{}Create the matrix K}
            K \PY{o}{=} \PY{k+kp}{diag}\PY{p}{(}m\PY{l+m}{+1}\PY{p}{)}
            K \PY{o}{=} \PY{o}{\PYZhy{}}K\PY{p}{[}\PY{l+m}{1}\PY{o}{:}m\PY{p}{,}\PY{p}{]}\PY{o}{+}K\PY{p}{[}\PY{l+m}{2}\PY{o}{:}\PY{p}{(}m\PY{l+m}{+1}\PY{p}{)}\PY{p}{,}\PY{p}{]}
            K \PY{o}{=} K\PY{p}{[}\PY{l+m}{1}\PY{o}{:}m\PY{p}{,}\PY{l+m}{1}\PY{o}{:}m\PY{l+m}{+1}\PY{p}{]}
            
            \PY{c+c1}{\PYZsh{}Return the product}
            M\PY{o}{\PYZpc{}*\PYZpc{}}K
        \PY{p}{\PYZcb{}}
\end{Verbatim}

    We also used - without declaration - a function ``linear combination'',
that should return the linear combination inside the inverse link

    \begin{Verbatim}[commandchars=\\\{\}]
{\color{incolor}In [{\color{incolor}13}]:} linear\PYZus{}combination \PY{o}{=} \PY{k+kr}{function}\PY{p}{(}X\PY{p}{,}Y\PY{p}{,}\PY{k+kp}{beta}\PY{p}{,}psi\PY{p}{)}\PY{p}{\PYZob{}}
             m \PY{o}{=} \PY{k+kp}{length}\PY{p}{(}psi\PY{p}{)} \PY{c+c1}{\PYZsh{}possible results, from 1 to m.}
             n \PY{o}{=} \PY{k+kp}{length}\PY{p}{(}Y\PY{p}{)} \PY{c+c1}{\PYZsh{}number of observatons}
             \PY{k+kr}{if}\PY{p}{(}\PY{k+kc}{TRUE}\PY{p}{)}\PY{p}{\PYZob{}}
                 beta2 \PY{o}{=} \PY{k+kt}{matrix}\PY{p}{(}\PY{k+kp}{rep}\PY{p}{(}\PY{k+kp}{beta}\PY{p}{,}m\PY{p}{)}\PY{p}{,}ncol\PY{o}{=}m\PY{p}{,}byrow\PY{o}{=}\PY{l+m}{0}\PY{p}{)}
             \PY{p}{\PYZcb{}}
             \PY{k+kp}{else}\PY{p}{\PYZob{}}
                 beta2 \PY{o}{=} \PY{k+kp}{beta}
             \PY{p}{\PYZcb{}}
             linear \PY{o}{=}  \PY{o}{\PYZhy{}} \PY{p}{(}X\PY{o}{\PYZpc{}*\PYZpc{}}beta2\PY{p}{)}
             origin \PY{o}{=}  \PY{k+kt}{matrix}\PY{p}{(}\PY{k+kp}{rep}\PY{p}{(}psi\PY{p}{,}n\PY{p}{)} \PY{p}{,}nrow \PY{o}{=} n\PY{p}{,} byrow \PY{o}{=} \PY{l+m}{1}\PY{p}{)}
             origin \PY{o}{+}  linear
         \PY{p}{\PYZcb{}}
\end{Verbatim}

    \begin{Verbatim}[commandchars=\\\{\}]
{\color{incolor}In [{\color{incolor} }]:} 
\end{Verbatim}


    % Add a bibliography block to the postdoc
    
    
    
    \end{document}
